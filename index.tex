% Options for packages loaded elsewhere
\PassOptionsToPackage{unicode}{hyperref}
\PassOptionsToPackage{hyphens}{url}
\PassOptionsToPackage{dvipsnames,svgnames,x11names}{xcolor}
%
\documentclass[
  letterpaper,
  DIV=11,
  numbers=noendperiod]{scrreprt}

\usepackage{amsmath,amssymb}
\usepackage{lmodern}
\usepackage{iftex}
\ifPDFTeX
  \usepackage[T1]{fontenc}
  \usepackage[utf8]{inputenc}
  \usepackage{textcomp} % provide euro and other symbols
\else % if luatex or xetex
  \usepackage{unicode-math}
  \defaultfontfeatures{Scale=MatchLowercase}
  \defaultfontfeatures[\rmfamily]{Ligatures=TeX,Scale=1}
\fi
% Use upquote if available, for straight quotes in verbatim environments
\IfFileExists{upquote.sty}{\usepackage{upquote}}{}
\IfFileExists{microtype.sty}{% use microtype if available
  \usepackage[]{microtype}
  \UseMicrotypeSet[protrusion]{basicmath} % disable protrusion for tt fonts
}{}
\makeatletter
\@ifundefined{KOMAClassName}{% if non-KOMA class
  \IfFileExists{parskip.sty}{%
    \usepackage{parskip}
  }{% else
    \setlength{\parindent}{0pt}
    \setlength{\parskip}{6pt plus 2pt minus 1pt}}
}{% if KOMA class
  \KOMAoptions{parskip=half}}
\makeatother
\usepackage{xcolor}
\setlength{\emergencystretch}{3em} % prevent overfull lines
\setcounter{secnumdepth}{5}
% Make \paragraph and \subparagraph free-standing
\ifx\paragraph\undefined\else
  \let\oldparagraph\paragraph
  \renewcommand{\paragraph}[1]{\oldparagraph{#1}\mbox{}}
\fi
\ifx\subparagraph\undefined\else
  \let\oldsubparagraph\subparagraph
  \renewcommand{\subparagraph}[1]{\oldsubparagraph{#1}\mbox{}}
\fi

\usepackage{color}
\usepackage{fancyvrb}
\newcommand{\VerbBar}{|}
\newcommand{\VERB}{\Verb[commandchars=\\\{\}]}
\DefineVerbatimEnvironment{Highlighting}{Verbatim}{commandchars=\\\{\}}
% Add ',fontsize=\small' for more characters per line
\usepackage{framed}
\definecolor{shadecolor}{RGB}{241,243,245}
\newenvironment{Shaded}{\begin{snugshade}}{\end{snugshade}}
\newcommand{\AlertTok}[1]{\textcolor[rgb]{0.68,0.00,0.00}{#1}}
\newcommand{\AnnotationTok}[1]{\textcolor[rgb]{0.37,0.37,0.37}{#1}}
\newcommand{\AttributeTok}[1]{\textcolor[rgb]{0.40,0.45,0.13}{#1}}
\newcommand{\BaseNTok}[1]{\textcolor[rgb]{0.68,0.00,0.00}{#1}}
\newcommand{\BuiltInTok}[1]{\textcolor[rgb]{0.00,0.23,0.31}{#1}}
\newcommand{\CharTok}[1]{\textcolor[rgb]{0.13,0.47,0.30}{#1}}
\newcommand{\CommentTok}[1]{\textcolor[rgb]{0.37,0.37,0.37}{#1}}
\newcommand{\CommentVarTok}[1]{\textcolor[rgb]{0.37,0.37,0.37}{\textit{#1}}}
\newcommand{\ConstantTok}[1]{\textcolor[rgb]{0.56,0.35,0.01}{#1}}
\newcommand{\ControlFlowTok}[1]{\textcolor[rgb]{0.00,0.23,0.31}{#1}}
\newcommand{\DataTypeTok}[1]{\textcolor[rgb]{0.68,0.00,0.00}{#1}}
\newcommand{\DecValTok}[1]{\textcolor[rgb]{0.68,0.00,0.00}{#1}}
\newcommand{\DocumentationTok}[1]{\textcolor[rgb]{0.37,0.37,0.37}{\textit{#1}}}
\newcommand{\ErrorTok}[1]{\textcolor[rgb]{0.68,0.00,0.00}{#1}}
\newcommand{\ExtensionTok}[1]{\textcolor[rgb]{0.00,0.23,0.31}{#1}}
\newcommand{\FloatTok}[1]{\textcolor[rgb]{0.68,0.00,0.00}{#1}}
\newcommand{\FunctionTok}[1]{\textcolor[rgb]{0.28,0.35,0.67}{#1}}
\newcommand{\ImportTok}[1]{\textcolor[rgb]{0.00,0.46,0.62}{#1}}
\newcommand{\InformationTok}[1]{\textcolor[rgb]{0.37,0.37,0.37}{#1}}
\newcommand{\KeywordTok}[1]{\textcolor[rgb]{0.00,0.23,0.31}{#1}}
\newcommand{\NormalTok}[1]{\textcolor[rgb]{0.00,0.23,0.31}{#1}}
\newcommand{\OperatorTok}[1]{\textcolor[rgb]{0.37,0.37,0.37}{#1}}
\newcommand{\OtherTok}[1]{\textcolor[rgb]{0.00,0.23,0.31}{#1}}
\newcommand{\PreprocessorTok}[1]{\textcolor[rgb]{0.68,0.00,0.00}{#1}}
\newcommand{\RegionMarkerTok}[1]{\textcolor[rgb]{0.00,0.23,0.31}{#1}}
\newcommand{\SpecialCharTok}[1]{\textcolor[rgb]{0.37,0.37,0.37}{#1}}
\newcommand{\SpecialStringTok}[1]{\textcolor[rgb]{0.13,0.47,0.30}{#1}}
\newcommand{\StringTok}[1]{\textcolor[rgb]{0.13,0.47,0.30}{#1}}
\newcommand{\VariableTok}[1]{\textcolor[rgb]{0.07,0.07,0.07}{#1}}
\newcommand{\VerbatimStringTok}[1]{\textcolor[rgb]{0.13,0.47,0.30}{#1}}
\newcommand{\WarningTok}[1]{\textcolor[rgb]{0.37,0.37,0.37}{\textit{#1}}}

\providecommand{\tightlist}{%
  \setlength{\itemsep}{0pt}\setlength{\parskip}{0pt}}\usepackage{longtable,booktabs,array}
\usepackage{calc} % for calculating minipage widths
% Correct order of tables after \paragraph or \subparagraph
\usepackage{etoolbox}
\makeatletter
\patchcmd\longtable{\par}{\if@noskipsec\mbox{}\fi\par}{}{}
\makeatother
% Allow footnotes in longtable head/foot
\IfFileExists{footnotehyper.sty}{\usepackage{footnotehyper}}{\usepackage{footnote}}
\makesavenoteenv{longtable}
\usepackage{graphicx}
\makeatletter
\def\maxwidth{\ifdim\Gin@nat@width>\linewidth\linewidth\else\Gin@nat@width\fi}
\def\maxheight{\ifdim\Gin@nat@height>\textheight\textheight\else\Gin@nat@height\fi}
\makeatother
% Scale images if necessary, so that they will not overflow the page
% margins by default, and it is still possible to overwrite the defaults
% using explicit options in \includegraphics[width, height, ...]{}
\setkeys{Gin}{width=\maxwidth,height=\maxheight,keepaspectratio}
% Set default figure placement to htbp
\makeatletter
\def\fps@figure{htbp}
\makeatother

\KOMAoption{captions}{tableheading}
\makeatletter
\makeatother
\makeatletter
\@ifpackageloaded{bookmark}{}{\usepackage{bookmark}}
\makeatother
\makeatletter
\@ifpackageloaded{caption}{}{\usepackage{caption}}
\AtBeginDocument{%
\ifdefined\contentsname
  \renewcommand*\contentsname{Table of contents}
\else
  \newcommand\contentsname{Table of contents}
\fi
\ifdefined\listfigurename
  \renewcommand*\listfigurename{List of Figures}
\else
  \newcommand\listfigurename{List of Figures}
\fi
\ifdefined\listtablename
  \renewcommand*\listtablename{List of Tables}
\else
  \newcommand\listtablename{List of Tables}
\fi
\ifdefined\figurename
  \renewcommand*\figurename{Figure}
\else
  \newcommand\figurename{Figure}
\fi
\ifdefined\tablename
  \renewcommand*\tablename{Table}
\else
  \newcommand\tablename{Table}
\fi
}
\@ifpackageloaded{float}{}{\usepackage{float}}
\floatstyle{ruled}
\@ifundefined{c@chapter}{\newfloat{codelisting}{h}{lop}}{\newfloat{codelisting}{h}{lop}[chapter]}
\floatname{codelisting}{Listing}
\newcommand*\listoflistings{\listof{codelisting}{List of Listings}}
\makeatother
\makeatletter
\@ifpackageloaded{caption}{}{\usepackage{caption}}
\@ifpackageloaded{subcaption}{}{\usepackage{subcaption}}
\makeatother
\makeatletter
\@ifpackageloaded{tcolorbox}{}{\usepackage[many]{tcolorbox}}
\makeatother
\makeatletter
\@ifundefined{shadecolor}{\definecolor{shadecolor}{rgb}{.97, .97, .97}}
\makeatother
\makeatletter
\makeatother
\ifLuaTeX
  \usepackage{selnolig}  % disable illegal ligatures
\fi
\IfFileExists{bookmark.sty}{\usepackage{bookmark}}{\usepackage{hyperref}}
\IfFileExists{xurl.sty}{\usepackage{xurl}}{} % add URL line breaks if available
\urlstyle{same} % disable monospaced font for URLs
\hypersetup{
  pdftitle={Tayyip Sinan Ocaktan's Progress Journal},
  colorlinks=true,
  linkcolor={blue},
  filecolor={Maroon},
  citecolor={Blue},
  urlcolor={Blue},
  pdfcreator={LaTeX via pandoc}}

\title{Tayyip Sinan Ocaktan's Progress Journal}
\author{}
\date{}

\begin{document}
\maketitle
\ifdefined\Shaded\renewenvironment{Shaded}{\begin{tcolorbox}[boxrule=0pt, interior hidden, enhanced, frame hidden, borderline west={3pt}{0pt}{shadecolor}, breakable, sharp corners]}{\end{tcolorbox}}\fi

\renewcommand*\contentsname{Table of contents}
{
\hypersetup{linkcolor=}
\setcounter{tocdepth}{2}
\tableofcontents
}
\bookmarksetup{startatroot}

\hypertarget{introduction}{%
\chapter*{Introduction}\label{introduction}}
\addcontentsline{toc}{chapter}{Introduction}

This progress journal covers Tayyip Sinan Ocaktan's work during their
term at \href{https://mef-bda503.github.io/fall22/}{BDA 503 Fall 2022}.

Each section is an assignment or an individual work.

\bookmarksetup{startatroot}

\hypertarget{assignment-1}{%
\chapter{Assignment 1}\label{assignment-1}}

Sinan Ocaktan\\
21 October 22

\hfill\break

Greetings! My name is Sinan. I have a BSc. in geomatics engineering from
ITU yet I have been working as a data scientist for over a year now. I
stumbled into machine learning when I was trying to calculate a missing
part of some quantitative data for a school project of mine. I still
remember what I googled: ``create an equation with known x and ys''.
Turns out my super \textbf{\emph{original}} idea was already there for
decades! I fell in love with the field and the community of it, so I am
here! Find more about me on my
\href{https://www.linkedin.com/in/ocaktans/?locale=en_US}{LinkedIn
page.}

\hypertarget{user-2022}{%
\section{UseR-2022}\label{user-2022}}

\hypertarget{emil-hvitfeldt---improvements-in-text-preprocessing-using-textrecipes}{%
\subsection{Emil Hvitfeldt - Improvements in text preprocessing using
`textrecipes`}\label{emil-hvitfeldt---improvements-in-text-preprocessing-using-textrecipes}}

The speaker promotes the newest improvements in
\href{https://cran.r-project.org/web/packages/textrecipes/index.html}{textrecipes
library}. textrecipes is a library that helps with text processing for
NLP. The speech consists of 3 main capabilities of the library:
tokenization, modification and numbers.

\hypertarget{tokenization}{%
\subsubsection{Tokenization}\label{tokenization}}

\includegraphics{https://www.simplilearn.com/ice9/free_resources_article_thumb/Tokenization.png}

Using raw texts is not a great way to talk with the computers. One of
the main processes in text processing is using tokens. We take pieces of
large strings and store them in \emph{tokens} to extract information in
a more specific way and help computers to understand human language.
There are several tokenization methods in textrecipes; one can split to
texts in characters, words, sentences and bytes.

\hypertarget{modification}{%
\subsubsection{Modification}\label{modification}}

\includegraphics[width=1\textwidth,height=\textheight]{https://images.all-free-download.com/images/graphiclarge/mechanism_vector_illustration_with_gears_and_mechanical_icons_design_6824575.jpg}

We also need to modificate our texts to avoid misleading parts of them
and get a more explanatory information.

One of the modification methods is stemming which is a process to access
the root meaning of the words. Despite the differences in their
appearence, some words can contain the same meaning and we want them to
describe the same thing to a computer.

Another method is to remove the stopwords. There are some words such as
``and'', ``the'' can be repeated a lot in texts while not meaning a lot.
They are usually dropped in NLP projects.

Sometimes computers store same looking characters in different ways and
while a human can see the same thing, the 2 characters will mean 2
different things to a computer. This happens in characters such as
``İ'', ``ö'' etc. textrecipes get through these difficulties by text
normalization.

While a word may mean a thing, its' order and the word before or after
that can be important. One can use sentence tokenization to capture them
but it has its' own disadvantages. textrecipes uses n-grams to store
tokens in pairs and get these meanings without needing sentences.

\hypertarget{numbers}{%
\subsubsection{Numbers}\label{numbers}}

\includegraphics{https://visme.co/blog/wp-content/uploads/2019/11/bar-graph-header-wide.jpg}

There is one important issue left with our texts: They are still texts!
We want to transform our tokens into numbers so they will have numbers
to describe theirselves. We can count our tokens with step\_tf feature
of textrecipes. Great, now we now how many times they appeared in our
set, yet we may want to transform them into frequencies as well. We call
this way of measure as ``tf-idf''. textrecipes can also help us with
that.

\hypertarget{r-posts}{%
\section{R Posts}\label{r-posts}}

\hypertarget{kable}{%
\subsection{Kable}\label{kable}}

We talked about text a lot yet I'd like to go on with them.
\href{https://www.kaggle.com/datasets/eward96/harry-potter-and-the-philosophers-stone-script}{Here}
is a fun dataset which consists of dialogues in the first book of Harry
Potter.

\begin{verbatim}
  scene     character_name
1     1   Albus Dumbledore
2     1 Minerva McGonagall
3     1   Albus Dumbledore
4     1 Minerva McGonagall
5     1   Albus Dumbledore
6     1 Minerva McGonagall
                                                                   dialogue
1         I should have known that you would be here, Professor McGonagall.
2           Good evening, Professor Dumbledore. Are the rumours true Albus?
3                          I'm afraid so, Professor. The good, and the bad.
4                                                              And the boy?
5                                                   Hagrid is bringing him.
6 Do you think it wise to trust Hagrid with something as important as this?
\end{verbatim}

We can see our dataset here, yet it doesn't look easy on the eyes.
That's why I spared my first topic for ``kable'' function of ``knitr''
package. You can find about it and more in
\href{https://medium.com/@SportSciData/https-medium-com-collinsneil306-how-to-create-interactive-reports-with-r-markdown-part-i-4fa9df46cd9}{this
link}.

To summary, kable is a great tool to adjust the options for our tables.
A simple code like below will change the whole appearence.

\begin{Shaded}
\begin{Highlighting}[]
\FunctionTok{kable}\NormalTok{(}\FunctionTok{head}\NormalTok{(df))}
\end{Highlighting}
\end{Shaded}

\begin{longtable}[]{@{}
  >{\raggedleft\arraybackslash}p{(\columnwidth - 4\tabcolsep) * \real{0.0606}}
  >{\raggedright\arraybackslash}p{(\columnwidth - 4\tabcolsep) * \real{0.1919}}
  >{\raggedright\arraybackslash}p{(\columnwidth - 4\tabcolsep) * \real{0.7475}}@{}}
\toprule()
\begin{minipage}[b]{\linewidth}\raggedleft
scene
\end{minipage} & \begin{minipage}[b]{\linewidth}\raggedright
character\_name
\end{minipage} & \begin{minipage}[b]{\linewidth}\raggedright
dialogue
\end{minipage} \\
\midrule()
\endhead
1 & Albus Dumbledore & I should have known that you would be here,
Professor McGonagall. \\
1 & Minerva McGonagall & Good evening, Professor Dumbledore. Are the
rumours true Albus? \\
1 & Albus Dumbledore & I'm afraid so, Professor. The good, and the
bad. \\
1 & Minerva McGonagall & And the boy? \\
1 & Albus Dumbledore & Hagrid is bringing him. \\
1 & Minerva McGonagall & Do you think it wise to trust Hagrid with
something as important as this? \\
\bottomrule()
\end{longtable}

But what about if we like to change more. Column names can change and we
can allign values in the columns for a neater look.

\begin{Shaded}
\begin{Highlighting}[]
\FunctionTok{kable}\NormalTok{(}\FunctionTok{head}\NormalTok{(df), }\AttributeTok{col.names =} \FunctionTok{c}\NormalTok{(}\StringTok{"Scene Number"}\NormalTok{, }\StringTok{"Character"}\NormalTok{, }\StringTok{"The Dialogue"}\NormalTok{), }\AttributeTok{align =} \StringTok{"clr"}\NormalTok{)}
\end{Highlighting}
\end{Shaded}

\begin{longtable}[]{@{}
  >{\centering\arraybackslash}p{(\columnwidth - 4\tabcolsep) * \real{0.1308}}
  >{\raggedright\arraybackslash}p{(\columnwidth - 4\tabcolsep) * \real{0.1776}}
  >{\raggedleft\arraybackslash}p{(\columnwidth - 4\tabcolsep) * \real{0.6916}}@{}}
\toprule()
\begin{minipage}[b]{\linewidth}\centering
Scene Number
\end{minipage} & \begin{minipage}[b]{\linewidth}\raggedright
Character
\end{minipage} & \begin{minipage}[b]{\linewidth}\raggedleft
The Dialogue
\end{minipage} \\
\midrule()
\endhead
1 & Albus Dumbledore & I should have known that you would be here,
Professor McGonagall. \\
1 & Minerva McGonagall & Good evening, Professor Dumbledore. Are the
rumours true Albus? \\
1 & Albus Dumbledore & I'm afraid so, Professor. The good, and the
bad. \\
1 & Minerva McGonagall & And the boy? \\
1 & Albus Dumbledore & Hagrid is bringing him. \\
1 & Minerva McGonagall & Do you think it wise to trust Hagrid with
something as important as this? \\
\bottomrule()
\end{longtable}

There are more things that can be done with kable and similar packages
but we got what we need for now.

\hypertarget{counts}{%
\subsection{Counts}\label{counts}}

We stored our data in a data frame, took a look to our data and can work
with it now. I would like to count the appearences of the characters in
the book and working with a data frame helps us a lot to do so.
\href{https://www.datasciencemadesimple.com/groupby-count-in-r-2/}{This
link} gives cool tricks for this task.

\begin{Shaded}
\begin{Highlighting}[]
\NormalTok{df\_freq }\OtherTok{\textless{}{-}} \FunctionTok{aggregate}\NormalTok{(df}\SpecialCharTok{$}\NormalTok{character\_name, }\AttributeTok{by=}\FunctionTok{list}\NormalTok{(df}\SpecialCharTok{$}\NormalTok{character\_name), }\AttributeTok{FUN=}\NormalTok{length)}
\NormalTok{df\_freq }\OtherTok{\textless{}{-}}\NormalTok{ df\_freq[}\FunctionTok{order}\NormalTok{(}\SpecialCharTok{{-}}\NormalTok{df\_freq}\SpecialCharTok{$}\NormalTok{x),]}
\FunctionTok{kable}\NormalTok{(}\FunctionTok{head}\NormalTok{(df\_freq), }\AttributeTok{row.names =} \ConstantTok{FALSE}\NormalTok{, }\AttributeTok{col.names =} \FunctionTok{c}\NormalTok{(}\StringTok{"Character"}\NormalTok{, }\StringTok{"Dialogue Count"}\NormalTok{), }\AttributeTok{align =} \StringTok{"lc"}\NormalTok{)}
\end{Highlighting}
\end{Shaded}

\begin{longtable}[]{@{}lc@{}}
\toprule()
Character & Dialogue Count \\
\midrule()
\endhead
Harry Potter & 230 \\
Ron Weasley & 120 \\
Hermione Granger & 92 \\
Rubeus Hagrid & 81 \\
Minerva McGonagall & 31 \\
Albus Dumbledore & 24 \\
\bottomrule()
\end{longtable}

Not so surprisingly, Harry Potter has the most dialogue more than twice
of his follower: Ron Weasley.

\hypertarget{word-cloud}{%
\subsection{Word Cloud}\label{word-cloud}}

Finally, we will create a word cloud. It is not exactly a word cloud
rather a character cloud. Word clouds are used to visualize text and
they look great in my opinion.

\begin{Shaded}
\begin{Highlighting}[]
\NormalTok{library(wordcloud2)}
\NormalTok{cloud2=wordcloud2(data=df\_freq, size=1)}
\NormalTok{print(cloud2)}
\end{Highlighting}
\end{Shaded}




\end{document}
